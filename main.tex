\documentclass{amsart}
\usepackage[utf8]{inputenc}

\documentclass{amsart}
\usepackage{mathtools}
\usepackage{centernot}
\usepackage{stmaryrd}
\usepackage{esvect}
\usepackage{upgreek}
\usepackage{comment}
\usepackage{amsfonts}
\usepackage{physics}
\usepackage{tikz-cd}
\usepackage[utf8]{inputenc}
\usepackage[english]{babel}
\usepackage{graphicx}
\usepackage[margin=1in]{geometry}
\usepackage{mathtools}


% \usepackage[nottoc]{tocbibind}
\usepackage{datetime}
\usepackage{graphicx}
\usepackage[nointegrals]{wasysym}
\usepackage{mathtools}
\usepackage{tikz}
\usepackage{tikz-cd}
\usepackage{enumerate}
\usepackage{marginnote}
\usepackage{amsmath, amsfonts, amssymb, amsthm}
\usepackage{bbm}
% \usepackage{thm-autoref}
\usepackage{hyperref}
\usepackage{booktabs}
\hypersetup{
  colorlinks   = true, %Colours links instead of ugly boxes
  urlcolor     = blue, %Colour for external hyperlinks
  linkcolor    = blue, %Colour of internal links
  citecolor   = blue %Colour of citations
}

\usepackage[textwidth=25mm]{todonotes}

\newtheorem{thm}{Theorem}[section]
\newtheorem*{unthm}{Theorem}

\newtheorem{lem}[thm]{Lemma}
% \newcommand{\lemautorefname}{Lemma}

\newtheorem{claim}[thm]{Claim}
\newtheorem{prop}[thm]{Proposition}
\newtheorem{cor}[thm]{Corollary}
\newtheorem{conj}[thm]{Conjecture}

\theoremstyle{definition}
\newtheorem{example}{Example}

\theoremstyle{definition}
\newtheorem*{question}{Question}

\theoremstyle{remark}
\newtheorem*{rem}{Remark}

\theoremstyle{definition}
\newtheorem{defn}{Definition}[section]

\usepackage[top=2cm, bottom=2cm, inner=2cm, outer=4cm, marginparwidth=3.4cm]{geometry}

% \usepackage{gentium}
% \usepackage[euler-digits,small]{eulervm}
\usepackage{enumerate}
\usepackage{xcolor}

\usepackage{import}
\usepackage{xifthen}
\usepackage{pdfpages}
\usepackage{transparent}
\usepackage{cite}

\newcommand*{\incfig}[2][1]{%
    \def\svgscale{#1}
    \import{./images/}{#2.pdf_tex}
}

\newcommand{\A}{\mathfrak{A}}

\newcommand{\Z}{\mathbb{Z}}
\newcommand{\R}{\mathbb{R}}
\newcommand{\C}{\mathbb{C}}

\newtheorem{manualtheoreminner}{Proposition}
\newenvironment{manualprop}[1]{%
  \renewcommand\themanualtheoreminner{#1}%
  \manualtheoreminner
}{\endmanualtheoreminner}

\setlength{\parindent}{12 pt}

\title{Fall 2020 DRP Writeup}
\author{Lily Li}
\date{\today}

\begin{document}

\maketitle

\tableofcontents

\section{Introduction}
\label{sec:intro}

The Riemann-Roch theorem is a theorem describing the space of meromorphic functions on a compact complex Riemann surface. In this section, we will establish some notation and state the theorem. 


\section{Preliminaries on Divisors}
\begin{defn}[Divisors]

\end{defn}

For the purposes of this theorem, the following quantities appear in the equation. 

$g$: the genus of the Riemann surface. It's also the dimension of the 1-forms of the first kind on the surface.

$r(\mathfrak{A}^{-1})$: the dimension of the vector space $L(\mathfrak{A}^{-1})=\{f\in\script{K}(M); (f)\geq\mathfrak{A}^{-1}\}$. 

$i(\A)$: the index of specialty of the divisor $\A$. In other words, this is the dimension of the space of abelian differentials $\omega$ with $(\omega)\geq \A$.

deg$\A$: The degree of the divisor.

\section{The Theorem}
\begin{thm}
Let M be a compact Riemann surface of genus g and $\A$ an integral divisor on $M$. Then 
$$r(\A^{-1})=\deg \A -g+1+i(\A)$$
\end{thm}

\begin{proof}
We first assume that $\A$ is strictly integral. Write it as the following 
$$\A = P_1^{n_1}\cdots P_m^{n_m}$$
\end{proof}

\bibliographystyle{alpha}
\bibliography{references}

\end{document}